%
% Niniejszy plik stanowi przykład formatowania pracy magisterskiej na
% Wydziale MIM UW.  Szkielet użytych poleceń można wykorzystywać do
% woli, np. formatujac wlasna prace.
%
% Zawartosc merytoryczna stanowi oryginalnosiagniecie
% naukowosciowe Marcina Wolinskiego.  Wszelkie prawa zastrzeżone.
%
% Copyright (c) 2001 by Marcin Woliński <M.Wolinski@gust.org.pl>
% Poprawki spowodowane zmianami przepisów - Marcin Szczuka, 1.10.2004
% Poprawki spowodowane zmianami przepisow i ujednolicenie 
% - Seweryn Karłowicz, 05.05.2006
% Dodanie wielu autorów i tłumaczenia na angielski - Kuba Pochrybniak, 29.11.2016

% dodaj opcję [licencjacka] dla pracy licencjackiej
% dodaj opcję [en] dla wersji angielskiej (mogą być obie: [licencjacka,en])
\documentclass[licencjacka,en]{pracamgr}

% Dane magistranta:
% \autor{Adam Deryło, Adrian Hess, Magdalena Pałkus, Michał Skwarek}{34234234}

% Dane magistrantów:
\autor{Adam Deryło}{342007}
\autori{Adrian Hess}{342013}
\autorii{Magdalena Pałkus}{231023}
\autoriii{Michał Skwarek}{777321}
%\autoriv{Autor nr Cztery}{432145}
%\autorv{Autor nr Pięć}{342011}

\title{GPU acceleration of CCSDS Rice decoding}
\titlepl{Akcerleracja GPU dekodowania CCSDS Rice}

%\tytulang{An implementation of a difference blabalizer based on the theory of $\sigma$ -- $\rho$ phetors}

%kierunek: 
% - matematyka, informacyka, ...
% - Mathematics, Computer Science, ...
\kierunek{Computer Science}

% informatyka - nie okreslamy zakresu (opcja zakomentowana)
% matematyka - zakres moze pozostac nieokreslony,
% a jesli ma byc okreslony dla pracy mgr,
% to przyjmuje jedna z wartosci:
% {metod matematycznych w finansach}
% {metod matematycznych w ubezpieczeniach}
% {matematyki stosowanej}
% {nauczania matematyki}
% Dla pracy licencjackiej mamy natomiast
% mozliwosc wpisania takiej wartosci zakresu:
% {Jednoczesnych Studiow Ekonomiczno--Matematycznych}

% \zakres{Tu wpisac, jesli trzeba, jedna z opcji podanych wyzej}

% Praca wykonana pod kierunkiem:
% (podać tytuł/stopień imię i nazwisko opiekuna
% Instytut
% ew. Wydział ew. Uczelnia (jeżeli nie MIM UW))
\opiekun{Paweł Gora\\
  Faculty of Mathematics, Informatics and Mechanics, University of Warsaw\\
  }

% miesiąc i~rok:
\date{Jan 2023}

%Podać dziedzinę wg klasyfikacji Socrates-Erasmus:
\dziedzina{ 
%11.0 Matematyka, Informatyka:\\ 
%11.1 Matematyka\\ 
% 11.2 Statystyka\\ 
11.3 Informatics, Computer Science \\ 
%11.3 Informatyka\\ 
%11.4 Sztuczna inteligencja\\ 
%11.5 Nauki aktuarialne\\
%11.9 Inne nauki matematyczne i informatyczne
}

%Klasyfikacja tematyczna wedlug AMS (matematyka) lub ACM (informatyka)
\klasyfikacja{D. Software\\
  D.1.3. Concurrent Programming\\
  I.4.2. Compression (Coding) \\ }

% Słowa kluczowe:
\keywords{CCSDS Rice coding, GPU, Compute Unified Device Architecture (CUDA)}

% Tu jest dobre miejsce na Twoje własne makra i~środowiska:
\newtheorem{defi}{Definicja}[section]

% koniec definicji

\begin{document}
\maketitle 

%tu idzie streszczenie na strone poczatkowa
\begin{abstract}
    TBD
\end{abstract}

\tableofcontents
%\listoffigures
%\listoftables

\chapter*{Introduction}
\addcontentsline{toc}{chapter}{Introduction}
Deep learning applications require complex, multi-stage data processing pipelines that include loading, decoding, cropping, resizing, and many other augmentations. These data processing pipelines, which are currently executed on the CPU, have become a bottleneck, limiting the performance and scalability of training and inference. To address this bottleneck, The NVIDIA Data Loading Library (DALI) was conceived with a goal of offloading data loading and prepossessing to the GPU. \\

Nevertheless, in the case of more unusual data formats such as FITS, which is a standard for space observation data, there are currently no solutions that would alleviate this bottleneck by harnessing GPU acceleration. One of the main problems that riddles FITS data format use case in  machine learning is the widespread utilization of rather nonstandard RICE compression. This coding method became a standard as it well suited for running on FPGAs used onboard satellites, although less performant on more mainstream hardware architectures compared to alternative data compression methods. \\ 
That's why, motivated by a growing number of deep learning applications trained on images of space, our team approached the nontrivial problem of massive GPU parallelization of RICE coding and its implementation in the DALI library.  




\chapter{Key terms}\label{r:pojecia}

\section{FITS}

\section{RICE}

\chapter{GPU acceleration of CCSDS Rice decoding algorithm}\label{r:losers}

\section{Naive approach}







\begin{thebibliography}{99}
\addcontentsline{toc}{chapter}{Bibliography}

\bibitem[Bea65]{beaman} Juliusz Beaman, \textit{Morbidity of the Jolly
    function}, Mathematica Absurdica, 117 (1965) 338--9.


\end{thebibliography}

\end{document}


%%% Local Variables:
%%% mode: latex
%%% TeX-master: t
%%% coding: latin-2
%%% End:
